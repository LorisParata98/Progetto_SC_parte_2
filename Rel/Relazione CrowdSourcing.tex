\documentclass[a4paper,11pt]{report}
\usepackage[italian]{babel}
\usepackage[utf8]{inputenc}
\usepackage[total={170mm,267mm},top=15mm,bottom=15mm,left=21mm,right=21mm]{geometry}
\usepackage{graphicx}

\begin{document}

\begin{titlepage}
  \clearpage\thispagestyle{empty}
  \centering
  \vspace{1cm}
  {\normalsize Informatica - Area scientifica \\  Dipartimento di Scienze matematiche, informatiche e multimediali\\  Università di Udine \par}
  \vspace{3cm}
  {\Huge \textbf{Progetto di Social Computing \newline
  } \LARGE{Crowdsorcing con Crowd Frame  parte 1}
  }

  \vspace{4cm}
  {\Large  Parata Loris (144338) \\ Arzon Francesco (142439)\\ Dal Fabbro Lorenzo (142300)\\ }
  \vspace{12cm}
  {\normalsize Anno accademico 2020/2021}
  \pagebreak
\end{titlepage}

\tableofcontents{}
\pagebreak
\chapter{Crowd Frame}
\section{Introduzione}
Questo secondo progetto di Social Computing si suddivide in due parti, la fase di progettazione di un task di crowdsourcing e la fase di raccolta dati e analisi dei dati ottenuti da un piccolo gruppo di workers.
In questa relazione si documenta  la prima parte dell'esperimento di crowdsourcing, la progettazione del task richiesto dalla traccia.


\section{Progettazione del task}

\subsection{Obbiettivo del task}
La traccia richiede di creare tre HIT differenti riguardanti tre libri, per ognuno dei quali si devono individuare tre edizioni differenti. Ogni HIT contiene tre edizioni, una per ogni libro di riferimento, per le quali ogni worker deve rispondere a 6 dimensioni differenti, quattro imposte dalla traccia e due scelte dal gruppo di progetto.
\subsubsection{Lingua del task}
Abbiamo deciso di sviluppare l'intero progetto in lingua inglese, in modo da rendere il progetto il più realistico e simile ai task presenti sulla piattaforma di crowdsourcing di MTurk. Di conseguenza abbiamo scelto dei libri le cui edizioni sono scritte in lingua inglese.

\subsection{Creazione del task}
La creazione del task è stata effettuata mediante l'utilizzo dell'interfaccia grafica del framework \textbf{Crowd Frame} per la semplicità con cui si presenta la GUI, anche se sono stati riscontrati alcuni problemi riguardanti la visualizzazione delle dimensioni che presentavano l'opzione "giustificazione". Problematiche che sono state corretti andando ad effettuare delle modifiche nel json delle dimensioni. \\

\begin{figure}[h]
	\centering
	\includegraphics[width=0.8\linewidth]{Relazione/Relazione/api_show_friendships}
	\label{fig:apishowfriendships}
\end{figure}

\subsection{Le istruzioni inerenti al task}
Le istruzioni... 


\begin{figure}[h]
	\centering
	\label{fig:followers}
\end{figure}

\subsection{Questionario}
Il questionario è composto da sei domande a risposta chiusa tradizionali che sono le seguenti:
\begin{itemize}
	\item 
	\item 
	\item 
	\item 
	\item 
	\item 
\end{itemize}

\subsection{Le dimensioni}
Oltre alle quattro dimensioni imposte dalla traccia:
\begin{itemize}
	\item 
	\item 
	\item 
	\item 
\end{itemize}
Abbiamo deciso di aggiungere :
\begin{itemize}
	\item 
	\item 
\end{itemize}


\subsection{Struttura della HIT}
Ogni HIT è composta da : \\



\section{Conclusioni}


\end{document}\\